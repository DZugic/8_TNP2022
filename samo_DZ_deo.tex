\section{Ciljevi digitalizacije}
\label{sec:naslovN}

Ciljevi digitalizcije se razlikuju u zavisnosti od toga zašto se primenjuje, želi li se
digitalizacijom zaštititi izvorno gradivo ili omogućiti pristup gradivu. S obzirom na današnju tehnologiju, dolazi se u mogućnost „izrade“ sadržaja kome se može bolje i
jednostavnije pristupiti kroz deljenje sadržaja institucije preko interneta s mnogobrojnim korisnicima istovremeno. Korisnici očekuju aktivno vođenje stranica ustanova i društvenih
mreža kako bi mogli da preslušaju određeni zvučni sadržaj bez fizičkog dolaska u ustanovu.
 
Ustanove digitalizuju svoje gradivo iz sledećih razloga:

\begin{itemize}
  \item kao plaćenu uslugu,
  \item kako bi dodale novi sadržaj u već postojeću zbirku (npr. priče, legende, bajke i sl.),
  \item vezano uz dokumentacijske procese,
  \item vezano uz internet stranicu
  \item radi konzervacije i restauracije,
  \item zbog ulaganja u nove tehnologije.
\end{itemize}

\section{Dobrobiti digitalizacije}
\label{sec:naslovM}

Digitalizacija se danas koristi sve više. Postaje skoro nepojmljivo da određena prodavnica nema bar internet stranicu, dok je to obavezni minimum koji se očekuje od većih kompanija i državnih institucija. Od elektronskih dnevnika, elektronskog bankarstva, društvenih mreža pa sve do muzočkih biblioteka sa milionima pesama, posledice digitalizacije vidimo na svakom koraku. 
Digitalizacija se koristi sve više. Postaje skoro nepojmljivo da određena prodavnica nema bar internet stranicu, dok je to obavezni minimum koji se očekuje od većih kompanija i državnih institucija. Od elektronskih dnevnika, elektronskog bankarstva, društvenih mreža pa sve do muzočkih biblioteka sa milionima pesama, posledice digitalizacije vidimo na svakom koraku. 

Pre digitalizacije, mnogi podaci, informacije, sadržaji su bili znatno skuplji, teže pristupačni, ograničenog broja i upotrebe. Ako bismo hteli da odslušamo novi album odrešenog izvošača, računar ili telefon je sve što nam je potrebno. Za veliki broj slučajeva, ne bismo morali da platimo, ne bismo imali uslov koliko puta smemo i koliko nas sme da sluša taj sadržaj. 
 
Vrlo bitan aspekt digitalizacije je katalogizacija. Zahvaljujući pravilnom arhiviranju i indeksiranju, imamo jasniji pregled i razumevanje sadržaja koje nam štedi vreme pri potrazi za odredjenim podacima, tokom obrade i rukovođenjem.
\section{Zaključak}
\label{sec:zakljucak}

Zvuk su mehaničke vibracije, a čovek prosečnog sluha može da registruje iz intervala 16 - 20 000 Hz. Početkom 19. veka su nastali analogne mašine kojima se mogao zabeležiti zvuk, od kojih je jedan primer gramofon. Upotreba gramofona nije bila preterano skupa, medjutim, ploče preko kojih se čuvao zvuk su bili krhki, osetljivi na temperaturu i tokom svakog koriščenja, štetio se reljef i gubili su se podaci. Nakon gramofona, dolazi do izuma magnetne trake, zatim kompaktne trake i konačno optičkih diskova. Optički diskovi se zarayziku od prethodno navedenih tehnologija nisu habali tokom reprodukcije i primene, al je moglo doći do grebanja tokom držanja i neobazrivog rukovođenja.

Zvučni zapisi sa analognih nosača se mogu  zaštititi od propadanja i učiniti dostupnim većem broju korisnika procesom digitalizacije. Digitalizovane zapise možemo spremiti u više
formata, a najpoznatiji su MP3, AIFF i WAV. Razlika među zapisima je u kvaliteti zapisa. WAV i AIFF su nekompresovani i kvalitetniji upravo jer ne dolzi do gubljenja podataka, kao kod kompresovanog MP3 zapisa. Nadalje, kako bi digitalizovani
zvuk bio zadovoljavajućeg kvaliteteta, potrebno je paziti na frekvenciju uzorkovanja, koja mora biti dvostruko veća od najviše frekvencije kako ne bi došlo do naoštravanja signala. Jedna metoda koja se koristi kako bi se to izbeglo je korišćenje antialiasing filtera, odnosno nisko
propusnog filtera kojim se „odrezuju“ visoke frekvencije. Kvantizacija je isto tako važan deo digitalizacije, pa i od nje zavisi kvalitet zvučnog zapisa. Važan je pojam dubina bita jer što je broj bitova veći, veći je i kvalitet.

Potrebno je digitalizirati gradivo, pogotovo kulturnu baštinu kako se identitet i kultura ne bi zaboravili i kako bi ljudi, odnosno korisnici, digitalizirano gradivo mogli da koriste i tokom celog života. S obzirom na današnju tehnologiju potrebno je i održavati gradivo u određenim formatima kako bi bili čitljivi i dostupni većem broju ljudi.
Analogni izvornici ili izvorno digitalno gradivo će uvek imati svoju vrednost u pogledu informacija.
