\documentclass[11pt]{article}
\usepackage[english,serbian]{babel}
\usepackage{graphicx}
\usepackage{epsfig}
\begin{document}

\section{Digitalizacija}
        MP3 format jedan je od najpoznatijih formata, a ostali poznatiji su WAV, AIFF, AAC,
        OGG i dr. MP3 format dolazi sa odredjenim gubicima i to je njegova najveća mana ali u isto vreme ima tu prednost manje veličine. Veličina zapisa bude gotovo 1/10 od nekompresovanog zapisa, kao sto su WAV ili AIFF formati koji ne zrtvuju nikoju količinu informacija zarad kompresije. Kvalitet
        MP3 formata zavisi od broja bitova koji će se koristiti, što je veći broj bitova, to je bolji zapis,
        bolji kvalitet. Nasuprot tome, postoje nekompresovani WAV i AIFF formati. To je WAV
        standardni format za snimanje zvučnih zapisa. Koriste se standardne i minimalne
        vrednosti – 44,1 kHz, 16-bitni ekran i dva kanala. AIFF format je veoma sličan WAV formatu.
        S obzirom na to da su oba formata nekompresovana, daje zvuk koji je bogat raznolikošću
        frekvencije, dobijaju se i originalno uzorkovanje i dubina pa su iz navedenih razloga
        lakše za dalju obradu. Njihov najveći nedostatak je veličina, koja može biti između 20-40 Mb po
        zapisu i ponekad nemogućnost razmene zapisa upravo zbog veličine.
        
     
  
            \begin{center}
     
        \begin{tabular}{lll} \hline
        & Nekompresovan WAV & Kompresovan MP3 \\ \hline
        41.4 kHz, 16-bit(CD) & 105.84 MB & 24 MB \\ \hline
        41.4 kHz, 32-bit & 211.68 MB & 46.5 MB \\ \hline
        41.4 kHz, 64-bit & 423.36 MB & 93 MB
        \end{tabular} 
            \\Veličine formata u trajanju od 10 minuta
            \end{center}
        
        Faze projekta digitalizacije odnose se na projekte digitalizacije koje realizuju institucije kao npr šta su arhivi, biblioteke ili muzeji i moraju uzeti u obzir određene resurse: ljudske, finansijsko, tehnološko i nasleđe. Pomenute institucije digitalizuju prvenstveno da bi stvarale digitalne kolekcije ili novi digitalni proizvodi. Faze u projektu digitalizacije su složene i zahtevne mnogo ljudskih napora jer zavise jedni od drugih: \\\\
        1. „definisanje vizije, svrhe i teme projekta,\\
        2. definisanje korisnika,\\
        3. istraživanje originalnog materijala,\\
        4. definisanje pristupa digitalizovanim materijalima,\\
        5. izbor materijala za digitalizaciju,\\
        6. provera autorskih prava i pregovaranje o licenciranju,\\
        7. pribavljanje finansijskih sredstava,\\
        8. primena standarda, smernica i primera dobre prakse,\\
        9. studija izvodljivosti, pilot projekat i stručno usavršavanje kadrova,\\
        10. priprema materijala za digitalizaciju,\\
        11. digitalno snimanje,\\
        12. kreiranje metapodataka,\\
        13. praćenje kvaliteta,\\
        14. projektovanje, proizvodnju i isporuku finalnog proizvoda,\\
        15. marketing,\\
        16. upravljanje projektima,\\
        17. dugoročno očuvanje digitalizovane građe.“ \\
        
Ciljevi digitalizacije su različiti u zavisnosti od toga zašto se digitalizuje, da li se želi
digitalizacijom zaštititi originalna građa ili da se omogući pristup materijalu njegovom digitalizacijom.
S obzirom na današnju tehnologiju, moguće je „napraviti“ strukturu koja može biti bolja i
lakši pristup i deljenje materijala ustanove putem interneta sa više korisnika istovremeno. Korisnici očekuju aktivno upravljanje institucionalnim i društvenim stranicama mreže kako bi mogli da slušaju određeni audio materijal bez fizičkog dolaska u ustanovu.
Institucije digitalizuju svoje materijale iz sledećih razloga: \\
• kao plaćena usluga, \\
• da se već postojećem fondu doda nova građa (npr. priče, legende, dijalekti itd.) \\
• u vezi sa procesima dokumentacije, \\
• vezano za veb lokaciju, \\
• radi očuvanja i restauracije,  \\
• zbog ulaganja u nove tehnologije. \\

Digitalizacija materijala se ne bi koristila da nema dobrih karakteristika koje u velikoj meri pomažu,
a neki od očiglednih su prenos opisanih i uređenih digitalnih zapisa korisnicima nezavisno od
udaljenost.
Digitalizacijom materijal postaje lakše i brže dostupan, odnosno ljudi bez dolaska u
institucije mogu koristiti materijal. Prilikom digitalizacije mora se i materijal urediti, što će mu olakšati kasnije pretraživanje, posebno ako je katalogizovan i indeksiran. Štaviše, oni će dobiti
pristup nedostupnim izvorima, npr. materijalu koji je zaštićen ili mu nikada nije bio i neće biti obezbeđen
koristit. Govoriti o materijalu koji je vredan i važan za ustanovu, a taj materijal nije dozvoljen
koriste zbog lošeg fizičkog stanja, pa će digitalne forme činiti takvu strukturu
dostupanom. Štaviše, digitalizovani materijal koji je dostupan korisnicima može se koristiti u
obrazovne svrhe, doživotno učenje i sticanje novih i zadržavanje postojećih znanja. Takođe, digitalizovanim materijalom se može lakše rukovati i može se koristiti lakše za obradu. Govoreći o audio materijalu, može se seći i spajati, filtrirati, ukloniti određene delove (klikove i druge šumove) itd. Digitalizacija materijala dovodi do zadovoljavanje potreba korisnika i smanjenje troškova. Jedna od najvažnijih i odličnih prednosti digitalizacije je ponovno sređivanje materijala. Tokom procesa digitalizacije građa se može dodatno restaurirati i sačuvati za budućnost da prave kolekcije većeg kvaliteta. \\

Digitalizovani materijal treba zaštititi od neovlašćenog pristupa, kopiranja i
distribucije, a njenu autentičnost mogu potvrditi mehanizmi zaštite građe. Neki
od metoda zaštite su: \\
• mehanizmi zaštite sistema, \\
• šifrovanje, \\
• digitalni potpisi, \\
• digitalni sertifikati, \\
• digitalni vodeni žigovi, \\
• šifrovane koverte.28 \\
Postoji mnogo mehanizama zaštite sistema i nijedan od njih nije savršen, a jedan od njih je češći
koristi se upravljanje nivoom pristupa. Prilikom uspostavljanja nivoa, to je sad omogućilo
pristup određenim podacima i uslugama, a pristup  nekim drugim je onemogućen. Za pristup se koriste lozinke koje treba da budu tajne i složene da ne mogu svi pristupiti podacima. Dalje, koriste se antivirusne zaštite, ali ih treba redovno održavati. Drugi način zaštite je zaštitni zid.

\end{document}