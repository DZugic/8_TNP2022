\documentclass{beamer}
\usepackage{beamerthemeshadow}
\usepackage{graphicx}
\usepackage{color}
\usepackage[utf8]{inputenc}
\usepackage{hyperref}
\usepackage[flushleft]{threeparttable}


\definecolor{UBCblue}{rgb}{0.03, 0.15, 0.4} % UBC Blue (primary)
\usecolortheme[named=UBCblue]{structure}

\def\d{{\fontencoding{T1}\selectfont\dj}}
\def\D{{\fontencoding{T1}\selectfont\DJ}}


\title{Tehničko i naučno pisanje}
\subtitle{-- Digitalizacija zvuka --}
\author{Nina Ostojić, Maksim Krstović, Mihailo Radulović, Dušan Žugić}
\institute{Matematički fakultet\\Univerzitet u Beogradu}
\date{
	\footnotesize{Beograd, 2022.}	
}

\begin{document}
\begin{frame}
	\thispagestyle{empty}
	\titlepage
\end{frame}

\addtocounter{framenumber}{-1}

\begin{frame}[fragile]\frametitle{Literatura}
	Zasnovano na:
	\begin{itemize}
		\item Jurković, Anamaria. Digitalizacija zvuka. Završni rad, Sveučilište u Zagrebu, Filozofski fakultet, 2021.
		\item S. Filipović. Digitalizacija zvuka. Računari i programiranje, Wordpress, 2014.
		\item Wordpress. Digital Sound and Music. on-line at: \url{http://digitalsoundandmusic.com/5-1-2-digitization/}
		\item Dr. Dheeraj Sanghi. Data Encoding. Computer Networks, Computer Science and Engineering, IIT Kanpur. on-line at: \url{https://www.cse.iitk.ac.in/users/dheeraj/cs425/lec03.html}
		\item Fizika 4, udžbenik za 4. razred gimnazije, Nataša Čaluković, Krug, Beograd
	\end{itemize}
\end{frame}

\begin{frame}
	\frametitle{Sadržaj} 
	\tableofcontents[hidesubsections] 
\end{frame}

\section{Uvod}
\subsection{Definicija zvuka}
\begin{frame}[fragile]\frametitle{Šta je zapravo zvuk?}
	\begin{itemize}	
 
		\item Zvuk je vibracija koja se širi kroz vazduh ili neku drugu sredinu u vidu talasa.
        \item Čovek može čuje zvuk jačine od 16 do 20000 Hz.
		\item Osnovne karakteristike zvuka su: 
		\begin{itemize}
			\item \textbf{visina}, 
			\item \textbf{boja}, 
			\item \textbf{jačina}.
		\end{itemize}
	\end{itemize}
\end{frame}

\subsection{Kratki istorijat zapisa zvuka}
\begin{frame}[fragile]\frametitle{Kako je sve počelo?}
	\begin{itemize}	
		\item Snimanje zvuka je počelo tako što su se zapisivale vibracije koje ljudsko uho registruje kao zvuk na neku vrstu medija.
		\item Fonoautograf
		\item Magnetna traka
        \item Kompaktna traka
		\item Početkom 1969. godine prvi put je uspešno realizovana računarska obrada zvuka koja je dovela do njegove digitalne transformacije.
		\item PCM procesori
		\item Kompaktni diskovi, CD-RW
		\item MP3
	\end{itemize}
\end{frame}

\section{Digitalizacija zvuka}
\subsection{Uzorkovanje}
\begin{frame}[fragile]\frametitle{Šta je digitalizacija?}
\begin{itemize}
	\item \textbf{Digitalizacija} je prevodenje analognog zvučnog signala u digitalni. Da bi se zvuk iz analognog preveo u digitalni oblik potrebno je izvršiti:
    \begin{itemize}			
		\item \textit{uzorkovanje ili odabiranje} (eng. \textit{sampling})
		\item \textit{kvantizaciju}
        \item \textit{kodiranje}
    \end{itemize}
	\item \textbf{Uzorkovanje} je postupak kojim se uzima vrednost električnog napona signala u odredenim trenucima vremena.
	\item Frekvencija uzorkovanja treba da bude najmanje dva puta veća od najveće frekvencije analognog signala (Nikvist-Šenonova teorema odabiranja).
	\item Standardna frekvencija uzorkovanja je 44,1kHz (CD audio standard), a može ići i do 192kHz. 
\end{itemize}

\end{frame}

\subsection{Kvantizacija i kodiranje}
\begin{frame}[fragile]\frametitle{Kvantizacija i kodiranje}
	\begin{itemize}	
		\item \textbf{Kvantizacija} je postupak kojim se odabrane vrednosti električnog napona zaokružuju na najbližu od dozvoljenih vrednosti.
		\item Dubina bita se prikazuje formulom: 2x = n, gde je x broj bitova, a n broj mogućih kombinacija.
		\item Kvantizacijska greška (eng. \textit{quantization error})
		\item \textbf{Kodiranje} (eng. \textit{Encoding}) je proces pretvaranja podataka u neki drugi format pri kojem se vrednosti predstavljaju logičkim nulama i jedinicama, tj. bitovima. 
		\item \textit{Pulsna kod modulacija} (PCM) i \textit{Delta modulacija} (DM)
		\item Bitska brzina (eng.\textit{bit rate})
	\end{itemize}
\end{frame}

\section{Kompresija i zaštitita}
\subsection{Čuvanje digitalnog zvuka}
\begin{frame}[fragile]\frametitle{Kompresija i čuvanje digitalnih podataka}
\begin{itemize}
  \item Za \textbf{čuvanje} je bitno odrediti koji se kvalitet medijuma traži: kapacitet, dugovečnost, pouzdanost, način rukovanja i naravno, cena kao i dostupnost. Uglavnom se koriste:
    \begin{itemize}
	 \item \textit{izmenjivi diskovi}
	 \item \textit{tvrdi diskovi}
	 \item \textit{magnetne trake}
    \end{itemize}
  \item \textbf{Kompresija} audio podataka ima potencijal da smanji propusni opseg prenosa i zahteve za skladištenje audio podataka.
  \item MP3 format jedan je od najpoznatijih formata. Kvalitet MP3 formata zavisi od broja bitova
  \item Nekompresovani WAV i AIFF formati. Ostali poznatiji su AAC, OGG i dr.
\end{itemize}
\end{frame}

\section{Ciljevi i dobrobiti digitalizacije}
\subsection{Zaštitita digitalizovanog materijala}
\begin{frame}[fragile]\frametitle{Zašto digitalizovati?}
	\begin{itemize}	
		\item Digitalizovani materijal treba zaštititi od neovlašćenog pristupa, kopiranja i distribucije. Neki od metoda zaštite su: 
		šifrovanje, digitalni potpisi, digitalni sertifikati, vodeni žigovi, šifrovane koverte itd 
		\item Ustanove digitalizuju svoju gradu iz više razloga:\\
		zbog lakšeg i dugotrajnog čuvanja, kako bi dodale novi sadržaj u već postojeću zbirku, vezano za administrativne procese, zbog izrade internet stranica, radi konzervacije i restauracije, zbog ulaganja u nove tehnologije.
		\item Elektronsko bankarstvo, elektronski dnevnici
		\item Društvene mreže
		\item Muzičke biblioteke i \textit{Streaming}
	\end{itemize}
\end{frame}

\section{Zakljucak}

\begin{frame}[fragile]\frametitle{Da rezimiramo}
	\begin{itemize}	
		\item Potrebno je digitalizovati gradu, pogotovo kulturnu baštinu, kako se identitet i kultura ne bi zaboravili i kako bi ljudi  
		digitalizovani materijal mogli da koriste i tokom celog života.
		\item S obzirom na današnju tehnologiju, potrebno je i održavati gradu u odredenim formatima kako bi bili čitljivi i dostupni većem broju ljudi.
		\item Analogni izvornici ili izvorni digitalni materijal će uvek imati svoju vrednost u pogledu informacija, pa i analogni zvuk, 
		jer koliko god pogodnosti digitalni zvuk pružao, naročito u smislu dugotrajnosti i uštede prostora, 
		postoji odredeni kvalitet 'živog' zvuka koji se neminovno gubi i koji je za sada nemoguće digitalno reprodukovati.
	\end{itemize}
	
\end{frame}

\end{document}
