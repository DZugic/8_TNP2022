\documentclass[11pt]{article}
\usepackage[english,serbian]{babel}
\usepackage{graphicx}
\usepackage{epsfig}
\begin{document}

\section{Digitalizacija}
        MP3 format jedan je od najpoznatijih formata, a ostali poznatiji su WAV, AIFF, AAC,
        OGG i dr. MP3 format dolazi sa odredjenim gubicima i to je njegova najveća mana ali u isto vreme ima tu prednost manje veličine. Veličina zapisa bude gotovo 1/10 od nekompresovanog zapisa, kao sto su WAV ili AIFF formati koji ne zrtvuju nikoju količinu informacija zarad kompresije. Kvalitet
        MP3 formata zavisi od broja bitova koji će se koristiti, što je veći broj bitova, to je bolji zapis,
        bolji kvalitet. Nasuprot tome, postoje nekompresovani WAV i AIFF formati. To je WAV
        standardni format za snimanje zvučnih zapisa. Koriste se standardne i minimalne
        vrednosti – 44,1 kHz, 16-bitni ekran i dva kanala. AIFF format je veoma sličan WAV formatu.
        S obzirom na to da su oba formata nekompresovana, daje zvuk koji je bogat raznolikošću
        frekvencije, dobijaju se i originalno uzorkovanje i dubina pa su iz navedenih razloga
        lakše za dalju obradu. Njihov najveći nedostatak je veličina, koja može biti između 20-40 Mb po
        zapisu i ponekad nemogućnost razmene zapisa upravo zbog veličine.
        
     
  
            \begin{center}
     
        \begin{tabular}{lll} \hline
        & Nekompresovan WAV & Kompresovan MP3 \\ \hline
        41.4 kHz, 16-bit(CD) & 105.84 MB & 24 MB \\ \hline
        41.4 kHz, 32-bit & 211.68 MB & 46.5 MB \\ \hline
        41.4 kHz, 64-bit & 423.36 MB & 93 MB
        \end{tabular} 
            \\Veličine formata u trajanju od 10 minuta
            \end{center}
        
Zašto kompresovati?
Kompresija se može koristiti za suptilno masiranje zvuka kako bi ona zvučala prirodnije i razumljivija bez dodavanja izobličenja, što rezultira pesmom koja je „udobnija“ za slušanje. Pored toga, mnogi kompresori - i hardverski i softverski - imaće prepoznatljiv zvuk koji se može koristiti za ubrizgavanje divnih boja i tonova u inače beživotne numere na primer. Isto tako, prekomerno kompresovanje određenog fajla može zaista uništiti ideju tog zvuka. Kompresija audio podataka, koju ne treba mešati sa kompresijom dinamičkog opsega, ima potencijal da smanji propusni opseg prenosa i zahteve za skladištenje audio podataka. Algoritmi audio kompresije su implementirani u softver kao audio kodeci. I kod kompresije sa gubicima i bez gubitaka, količina informacija je smanjena korišćenjem metoda kao što su kodiranje, kvantizacija, diskretna kosinusna transformacija i linearno predviđanje kako bi se smanjila količina informacija koje se koriste za predstavljanje nekompresovanih podataka.

Digitalizacija materijala se ne bi koristila da nema dobrih karakteristika koje u velikoj meri pomažu,
a neki od očiglednih su prenos opisanih i uređenih digitalnih zapisa korisnicima nezavisno od
udaljenost.
Digitalizacijom materijal postaje lakše i brže dostupan, odnosno ljudi bez dolaska u
institucije mogu koristiti materijal. Prilikom digitalizacije mora se i materijal urediti, što će mu olakšati kasnije pretraživanje, posebno ako je katalogizovan i indeksiran. Štaviše, oni će dobiti
pristup nedostupnim izvorima, npr. materijalu koji je zaštićen ili mu nikada nije bio i neće biti obezbeđen
koristit. Govoriti o materijalu koji je vredan i važan za ustanovu, a taj materijal nije dozvoljen
koriste zbog lošeg fizičkog stanja, pa će digitalne forme činiti takvu strukturu
dostupanom. Štaviše, digitalizovani materijal koji je dostupan korisnicima može se koristiti u
obrazovne svrhe, doživotno učenje i sticanje novih i zadržavanje postojećih znanja. Takođe, digitalizovanim materijalom se može lakše rukovati i može se koristiti lakše za obradu. Govoreći o audio materijalu, može se seći i spajati, filtrirati, ukloniti određene delove (klikove i druge šumove) itd. Digitalizacija materijala dovodi do zadovoljavanje potreba korisnika i smanjenje troškova. Jedna od najvažnijih i odličnih prednosti digitalizacije je ponovno sređivanje materijala. Tokom procesa digitalizacije građa se može dodatno restaurirati i sačuvati za budućnost da prave kolekcije većeg kvaliteta. \\

Digitalizovani materijal treba zaštititi od neovlašćenog pristupa, kopiranja i
distribucije, a njenu autentičnost mogu potvrditi mehanizmi zaštite građe. Neki
od metoda zaštite su: \\
• mehanizmi zaštite sistema, \\
• šifrovanje, \\
• digitalni potpisi, \\
• digitalni sertifikati, \\
• digitalni vodeni žigovi, \\
• šifrovane koverte. \\
Postoji mnogo mehanizama zaštite sistema i nijedan od njih nije savršen, a jedan od njih je češći
koristi se upravljanje nivoom pristupa. Prilikom uspostavljanja nivoa, to je sad omogućilo
pristup određenim podacima i uslugama, a pristup  nekim drugim je onemogućen. Za pristup se koriste lozinke koje treba da budu tajne i složene da ne mogu svi pristupiti podacima. Dalje, koriste se antivirusne zaštite, ali ih treba redovno održavati. Drugi način zaštite je zaštitni zid.

\end{document}